\documentclass[]{article}
\usepackage{lmodern}
\usepackage{amssymb,amsmath}
\usepackage{ifxetex,ifluatex}
\usepackage{fixltx2e} % provides \textsubscript
\ifnum 0\ifxetex 1\fi\ifluatex 1\fi=0 % if pdftex
  \usepackage[T1]{fontenc}
  \usepackage[utf8]{inputenc}
\else % if luatex or xelatex
  \ifxetex
    \usepackage{mathspec}
  \else
    \usepackage{fontspec}
  \fi
  \defaultfontfeatures{Ligatures=TeX,Scale=MatchLowercase}
\fi
% use upquote if available, for straight quotes in verbatim environments
\IfFileExists{upquote.sty}{\usepackage{upquote}}{}
% use microtype if available
\IfFileExists{microtype.sty}{%
\usepackage{microtype}
\UseMicrotypeSet[protrusion]{basicmath} % disable protrusion for tt fonts
}{}
\usepackage[margin=1in]{geometry}
\usepackage{hyperref}
\hypersetup{unicode=true,
            pdftitle={Be Practical!},
            pdfauthor={JB},
            pdfborder={0 0 0},
            breaklinks=true}
\urlstyle{same}  % don't use monospace font for urls
\usepackage{longtable,booktabs}
\usepackage{graphicx,grffile}
\makeatletter
\def\maxwidth{\ifdim\Gin@nat@width>\linewidth\linewidth\else\Gin@nat@width\fi}
\def\maxheight{\ifdim\Gin@nat@height>\textheight\textheight\else\Gin@nat@height\fi}
\makeatother
% Scale images if necessary, so that they will not overflow the page
% margins by default, and it is still possible to overwrite the defaults
% using explicit options in \includegraphics[width, height, ...]{}
\setkeys{Gin}{width=\maxwidth,height=\maxheight,keepaspectratio}
\IfFileExists{parskip.sty}{%
\usepackage{parskip}
}{% else
\setlength{\parindent}{0pt}
\setlength{\parskip}{6pt plus 2pt minus 1pt}
}
\setlength{\emergencystretch}{3em}  % prevent overfull lines
\providecommand{\tightlist}{%
  \setlength{\itemsep}{0pt}\setlength{\parskip}{0pt}}
\setcounter{secnumdepth}{0}
% Redefines (sub)paragraphs to behave more like sections
\ifx\paragraph\undefined\else
\let\oldparagraph\paragraph
\renewcommand{\paragraph}[1]{\oldparagraph{#1}\mbox{}}
\fi
\ifx\subparagraph\undefined\else
\let\oldsubparagraph\subparagraph
\renewcommand{\subparagraph}[1]{\oldsubparagraph{#1}\mbox{}}
\fi

%%% Use protect on footnotes to avoid problems with footnotes in titles
\let\rmarkdownfootnote\footnote%
\def\footnote{\protect\rmarkdownfootnote}

%%% Change title format to be more compact
\usepackage{titling}

% Create subtitle command for use in maketitle
\newcommand{\subtitle}[1]{
  \posttitle{
    \begin{center}\large#1\end{center}
    }
}

\setlength{\droptitle}{-2em}

  \title{Be Practical!}
    \pretitle{\vspace{\droptitle}\centering\huge}
  \posttitle{\par}
    \author{JB}
    \preauthor{\centering\large\emph}
  \postauthor{\par}
      \predate{\centering\large\emph}
  \postdate{\par}
    \date{2018-10-21}


\begin{document}
\maketitle

I thought I would change gears into something more practical this week,
delivering a basic analytic product end to end. I will preface this with
I hate doing analysis in Excel, but the world's most prolific software
isn't going anytime soon. To be fair, most people are comfortable with
it so that is what we will discuss in this post.

What I always recommend to people is they treat their analysis as thee
distinct parts. That doesn't mean you cannot go back iterate, but like
washing your hands when cooking you are trying to prevent cross
contamination and maintain order. The three-part rule allows you to
understand the clear purpose of your work so you know what success looks
like. The three parts include \textbf{Data Transformation},
\textbf{Analysis} and \textbf{Visualization}. Before you get going with
anything start at the visualization and ask yourself ``What will the
customer see?'', ``What have they asked for?'' and ``What is the best
way to present this information?'' That will dictate what analysis and
data you need. Do you want a time series? Then you need dates. Do you
want a count of categories? Then you need frequencies. The important
thing is what the `end' looks like for the customer.

\textbf{Data Transformation} In this step you would take you data and
rearrange it to suit your analysis. There are two types of tabulate
data, wide and long format. In additional all data in a table can be
thought of as a matrix (refered to as \(a\)) where \(i\) (the rows) and
\(j\) (the columns) so make a table, or matrix, or \(a_i,_j\).

The table below is an example of wide format data. It is wide because
one row index has repeated observations by columns. In this case,
country as an index has three observations of population between 2001
and 2003. Wide format is great for isolating a subject, such as country,
and looking at how things may change over time. Also known as repeated
measures analysis.

\begin{longtable}[]{@{}llll@{}}
\toprule
Country & Population 2001 & Population 2002 & Population
2003\tabularnewline
\midrule
\endhead
Canada & 31,021,000 & 31,373,000 & 31,676,000\tabularnewline
United States & 284,970,000 & 287,630,000 & 290,110,000\tabularnewline
United Kingdom & 59,113,016 & 59,365,677 & 59,636,662\tabularnewline
France & 59,549,549 & 59,893,849 & 60,270,320\tabularnewline
Germany & 82,350,000 & 82,490,000 & 82,530,000\tabularnewline
\bottomrule
\end{longtable}

Long format is slightly different. Note that we have now split the
column titles of population and year into separate columns. For most
analysis, we can now isolate that attribute and use filters to specify
population or year for a particular country. This is most commonly used
for summarizing data.

\begin{longtable}[]{@{}lll@{}}
\toprule
Country & Population & Year\tabularnewline
\midrule
\endhead
Canada & 2001 & 31,021,000\tabularnewline
Canada & 2002 & 31,373,000\tabularnewline
Canada & 2003 & 31,676,000\tabularnewline
United States & 2001 & 284,970,000\tabularnewline
United States & 2002 & 287,630,000\tabularnewline
United States & 2003 & 290,110,000\tabularnewline
United Kingdom & 2001 & 59,113,016\tabularnewline
United Kingdom & 2002 & 59,365,677\tabularnewline
United Kingdom & 2003 & 59,636,662\tabularnewline
France & 2001 & 59,549,549\tabularnewline
France & 2002 & 59,893,849\tabularnewline
France & 2003 & 60,270,320\tabularnewline
Germany & 2001 & 82,350,000\tabularnewline
Germany & 2002 & 82,490,000\tabularnewline
Germany & 2003 & 82,530,000\tabularnewline
\bottomrule
\end{longtable}

Since we are also dealing in the real world, you may have incomplete or
incorrect data such as , 0 population or the year 2099. When the happens
it is worth correcting the data if possible but most likely you can
simply mark it as missing. For numeric data use a character that you
likely would not need like `-99' and for categorical data make the data
as ``NA''.

\textbf{Analysis} Now that your data is looking spiffy, you need to
analyse it. The big question here is are you summarizing it (ie. Showing
what has happened in the past) or are you forecasting to predict the
future. One is easier than the other. Using past data we don't have to
worry about model fit and strength as we know what happened. Prediction
and forecasting require some higher levels of knowledge. So let's keep
it simple today and just go with summary. Basic graphs types for
analysis include:

Histogram- Shows the distribution of numeric data by a given values in a
variable as a frequency. It is important to remember that histograms
only take one variable at a time and usually measures the number of
items with a given values. This graph can be extended to include normal
distribution to see how your data makes an objective theoretical
dataset.

Bar Charts- Used to compare the frequency of variables together. You can
have two types of bar chart; stacked (where all the values are presented
as segments of the same bar) or grouped (similar to a histogram where a
bar is split by values within a variable or some other grouping
variable).

Line Chart- Draws a continuous line between points on a X and Y axis
where X is a continuous scale variable and Y is the measurement such as
population over time. Can be fitted with a line of best fit to project
the general direction of data for forecasting.

Scatterplot- Pointing independent points on a plane of X and Y where X
is a continuous scale variable and Y is the measurement as a pair (ie.
2001,31021000). Here the points are not joined but some relationship
between the X and Y point. Can also be fitted with a line of best fit to
project the general direction of data for forecasting.

\textbf{Visualization} After the whistle-stop tour of analysis we move
on to visualization. The above graphs bleed into this topic, but there
is still important work to be done. Do you have multiple graphs? Then
arrange them from the broad to the specific like you are telling a
story. If you use the same variable over again use the same colour so
the customers eye can quickly recognize it. Avoid a lot of text in
graphs unless necessary. Use simple names for all labels or axis. If you
have many categories avoid using rainbow colours as the graph will look
busy or try to reduce or separate the columns to make it more
manageable.

The beauty of the three-part rule means if a customer says ``I would
also like'' or ``can we also show'' you can quickly interpret their
needs and go back to the part of the analysis to be changed. If your
analysis and graphs are configured correctly, they should cascade any
changes automatically meaning you can quickly iterate on changes under
you have a nice shiny output.

It is always important to specify your three-parts during or after your
analysis by writing down what you did and how you did it. Don't add it
to your presentation as it will distract from the story you are telling.
If the customer wants it it will be there for them, but typically they
just want the graph. Having it written down in the three distinct parts
allows for a clean organizational flow to you method that makes
intuitive sense. A biproduct of writting your method down is that it
teaches and reinforces the principles of good analysis to customers who
are not data savvy. Not only will they get a top-notch product, but will
also learn a little about what goes on behind the scenes.

I hope this is helpful.What have your experiences been doing analysis? I
want to hear from folks about your trials, tribulations and triumphs
delivering analysis to customers. And if you can apply my method in your
next project, please give a shout out to let me know how you got on.


\end{document}
